% !TEX TS-program = pdflatex
% !TEX encoding = UTF-8 Unicode

\documentclass[11pt]{article} 

\usepackage[utf8]{inputenc}	% set input encoding
\usepackage{hyperref}		% clickable links
\usepackage{color}			% text coloring enabled
\usepackage{geometry}		% to change the page dimensions
\geometry{a4paper}		% or letterpaper (US) or a5paper or....
\usepackage[parfill]{parskip} 	% begin paragraphs with an empty line rather than an indent
\usepackage{amstext}


\begin{document}

   \framebox{
      \vbox{\vspace{2mm}
    \hbox to 6.28in { {\bf Introduction to Recommender Systems
		\hfill Session:} 3 Sep 2013 }
    \hbox to 6.28in {  Coursera \url{https://www.coursera.org/course/recsys}
		\hfill (14 weeks long) }
      \vspace{4mm}

    \hbox to 6.28in {  by Joseph A Konstan and Michael D Ekstrand
		\hfill {\color{cyan} {\small {\it Notes taken by} } }}
    \hbox to 6.28in {  {\it University of Minnesota}
		\hfill {\color{cyan} {\small \href{github.com/traims}{\it github.com/traims}}}}
      \vspace{2mm}}
   }



\section{Basic product association recommenders ~$^{Video~2.1}$}

Which product to buy with a product $X$? 
We consider \emph{non-personalized} recommendations,
which are the same for every user looking at a particular product.

\subsection{Looking for things that co-occur}

We know that a person bought $X$. Which product $Y$ to recommend?

The simplest way to compute the ranking of a product $Y$:

$\text{percentage of $X$-buyers who also bought $Y$ : }
\frac{\text{people who bought both $X$ and $Y$}}
{\text{people who bought $X$}}$

\emph{When it doesn't work}: in a supermarket, almost everyone buys bananas---that doesn't mean that we should recommend buying bananas to accompany every product.

\subsection{Looking where opinions diverge from the community average}


\end{document}
